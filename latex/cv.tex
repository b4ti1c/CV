\documentclass[11pt,a4paper,helvet]{moderncv}

\moderncvstyle{casual} % CV theme - options include: 'casual' (default), 'classic', 'oldstyle' and 'banking'
\moderncvcolor{blue} % CV color - options include: 'blue' (default), 'orange', 'green', 'red', 'purple', 'grey' and 'black'

\usepackage[scale=0.75]{geometry}
\usepackage{relsize}
%\usepackage{hyperref}


\firstname{A. Hazar} % Your first name
\familyname{\.Ilhan} % Your last name

\title{Curriculum Vitae}

\photo[70pt][0.4pt]{pictures/picture}

\begin{document}
\makecvtitle

\relscale{0.95}

\section{Personal}

\cvitem{Fullname}{Aytekin Hazar \.Ilhan}
\cvitem{Date of birth}{7\textsuperscript{th} November 1988}
\cvitem{Nationality}{Turkish}
\cvitem{Marital}{Single}
\cvitem{Military}{Postponed to 2016}
\cvitem{Languages}{
\begin{itemize}
\item Turkish -Native
\item English -Advanced
\end{itemize}
}
\cvitem{Contact}{
\begin{itemize}
\item Address
\begin{itemize}
\item H\"urriyet Mahallesi, Vatansever caddesi, G\"unbatimi sokak, Tip Mensuplari Sitesi No:35, Yakacik/Kartal, \.Istanbul/Turkey, Zipcode: 34876
\end{itemize}
\item Phone
\begin{itemize}
\item +90 533 273 49 03
\end{itemize}
\item Web
\begin{itemize}
\item E-mail: hazarilhan@sabanciuniv.edu
\item Homepage: \href{http://hazarilhan.com}{\textcolor{blue}{http://hazarilhan.com}} (under construction)
\item Linked-in: \href{ https://www.linkedin.com/pub/hazar-ilhan/65/aa8/259 }{\textcolor{blue}{ https://www.linkedin.com/pub/hazar-ilhan/65/aa8/259}}
\end{itemize}
\end{itemize}
}

\section{Education}

\cventry{2011--2014}{Master of Science}{Sabanci University}{Full-Scholarship (BIDEB)}{\textit{GPA -- 3.80}}{Computer Science \& Engineering
\newline{}
\textit{\textbf{Thesis:} Design and Implementation of a Digital Holographic Microscope with Fast Autofocusing}}

\cventry{2006--2011}{Bachelor of Science}{Sabanci University}{2/3 Scholarship}{\textit{GPA -- 3.25}}{Computer Science \& Engineering}

\newpage

\section{About Me}
I have always been intrigued by theoretical systems and their practical implications. Throughout my life, I had worked in teams with electronic engineers (EE), in which I humbly believe, led me to a greater understanding of what is possible with future's technology and a broad vision of what possibilities exist for men from today's perspective. I truely believe that `the singularity is near` and I'm all for innovation.


\section{Skills}

\cvitem{Cognitive}{
\begin{itemize}
\item Teamworker
\item Reasoning
\item Fast-adapter
\item Fast-learner
\item Innovative
\item Patient
\item Musician
\end{itemize}
}
\cvitem{Technical}{
\begin{itemize}
\item Conceptual
\begin{itemize}
\item Mobile \& Web technologies (Back end \& Front end)
\item Signal Processing, Image Processing
\item Parallel Programming (GP-GPU)
\item Optics (Digital Holography)
\item Artificial Intelligence
\item Embedded Platforms
\end{itemize}
\item Programming
\begin{itemize}
\item Javascript (ES5-6) - CSS3 - HTML5
\item NodeJS, AngularJS, TartJS, MVC and MVVM patterns
\item Matlab
\item Verilog
\item C/C++/C\#, Objective-C (
\end{itemize}
\item Tools
\begin{itemize}
\item Sublime
\item PhoneGap/Cordova
\item Google Closure
\item MongoDB
\item Electron
\item Git
\item Xilinx
\item LaTEX
\item Logic Pro, Pro Tools
\end{itemize}
\end{itemize}
}

\newpage

\section{Work Experience}

\cventry{2014--2015}{Kidstory}{Developer}{}{}{Kidstory is a start-up company, and it develops an application that listens your speech and simulataneously plays related music and/or sound effect while you are reading a story to your child. The application is currently only available in iOS. I developed both the front end \& the back end of this application. In 2015 Summer, founders decided to halt the development cycle and wanted to focus on the business model and marketing strategies. Further details can be found in \href{http://kidstory.co}{\textcolor{blue}{http://kidstory.co}}
\newline{}\newline{}
Tools used:
\begin{itemize}
\item TartJS
\item Google Closure
\item PhoneGap/Cordova
\item MongoDB
\item AngularJS (Admin panel)
\end{itemize}}

\cventry{2015 Summer}{Melodikapp}{Developer}{}{}{A simple PC application developed for Melodika Game Studio to partially-automate and significantly improve the efficiency of audio-recording sessions of multi-language games.
\newline{}\newline{}
Tools used:
\begin{itemize}
\item AngularJS
\item Electron
\end{itemize}}

\cventry{2013}{Traxio}{Contributing Developer}{}{}{Traxio is a free HTML5-based digital audio workstation that enables musicians to group-up from across the globe for recording, editing and mixing a song together simultaneously. The projects are auto-synched to cloud, and thus, songs can be accessed or mixed from any device. This project is open-source and still under development. You can check it out at: \href{http://trax.io/}{\textcolor{blue}{http://trax.io/}}
\newline{}\newline{}
Tools used:
\begin{itemize}
\item AngularJS
\item TartJS
\item WebAudio API
\item Google Closure
\item MongoDB
\end{itemize}}

\cventry{2010 Summer}{Intern at Scheidt \& Bachmann}{Developer}{}{}{Scheidt \& Bachmann is a well-established German company that builds systems for parking \& leisure centers, signalling, fare collection and petrol stations. During my 2 months internship, I had worked in the petrol stations department, and I was in charge of improving receipt generation according to the feedback from video camera. Due to the quick-completion of this task, I was assigned to 2 additional projects, which included improving the user interface of an on-field controller for company's technicians, and serialization of oil-tank level-data that arrived to the back-end through a RS-485 parallel peripheral.
\newline{}\newline{}
Tools used:
\begin{itemize}
\item C (Visual Studio)
\item Visual Basic
\end{itemize}}

\newpage
\section{Academic Experience}

\cventry{2011--2014}{Real-time DHM with Optical Tweezer}{Sabanci University}{Team-Developer}{}{Although licensed as Computer Science, my graduate education was involved with mainly Optics and Signal Processing. Through collaboration with an EE peer, we built a digital holographic microscope (DHM) and it's controlling software. This software utilizes an Nvidia GPU to perform intensive holographic calculations in real-time. In industry, DHMs are used for high-precision 3D imaging of objects and cells without staining. In our case, the DHM was incorporated with an optical tweezer and live cells were manipulated using the tweezer. My master's thesis benefits from this device as well.
\newline{}\newline{}
Tools used:
\begin{itemize}
\item Matlab, C++, C\#
\item Image Processing
\item Parallel Programming (CUDA)
\end{itemize}}

\cventry{2011--2013}{Teacher Assistant}{Sabanci University}{Supervisor}{}{I have performed assitantship for Logic (CS) and Microprocessors (EE) courses in my graduate education. Assistantship duties included supervising stu
dents during lab hours, preparing semester projects and reading \& evaluation of homeworks \& examinations.
\newline{}\newline{}
Tools used:
\begin{itemize}
\item Verilog
\item Xilinx
\item PicAssembly
\item ANSI-C
\end{itemize}}

\cventry{2010--2011}{AESwarm}{Sabanci University}{Team-Developer}{}{AESwarm is my graduation project with an EE peer to build low-cost intelligent robots that demonstrate swarm behaviour. The project was funded by Elektrik M\"uhendisleri Odasi (EMO). After running simulations, we designed and produced robots below the cost of a 50\$ which included two brushless dc motors, a dspic33 microprocessor, 8 infrared emitters \& sensors, a white-light sensor, a buzzer, and a polymer chasis. The boards were produced in Sabanci University laboratuaries and components were put together by ourselves. The robots made use of Self-Organizing Neural Networks to decide on an action depending on sensorial inputs. 
\newline{}\newline{}
Tools used:
\begin{itemize}
\item C\# \& OpenGL (Simulation)
\item ANSI-C
\item Eagle (Electronics board sketching software)
\item Artifical Intelligence (SOMs)
\end{itemize}}

\cventry{2010}{V-MIPS}{Sabanci University}{Team-Developer}{}{V-MIPS is a 16-bit mips processor that is able to process arrays of data at one cycle (similar to graphics processors) in 40Mhz. The processor consists of 5 pipelines and adheres to the standard mips instruction set. The processor is a full custom design implemented in Verilog and run on a Spartan-3 FPGA. We were able to drive VGA monitors while testing the performance of this processor.
\newline{}\newline{}
Tools used:
\begin{itemize}
\item Spartan-3 FPGA
\item Verilog
\item Xilinx
\end{itemize}}

\newpage

\begin{thebibliography}{}

\bibitem {}
Hazar A. \.Ilhan, Mert Do\v{g}ar, Meri\c{c} \"Ozcan (2014),
\newblock {\em Digital holographic microscopy and focusing methods based on image sharpness},
\newblock Journal of Microscopy, Volume: 255 Issue: 3, 138--149, 
\newblock \newline{} \href{http://dx.doi.org/10.1111/jmi.12144}{\textcolor{blue}{http://dx.doi.org/10.1111/jmi.12144}}

\bibitem {}
Mert Do\v{g}ar, Hazar A. \.Ilhan, Meri\c{c} \"Ozcan (2013),
\newblock {\em Real-time, auto-focusing digital holographic microscope using graphics processors},
\newblock Review of Scientific Instruments, Volume: 84, No: 8, 083704
\newblock \newline{} \href{http://dx.doi.org/10.1063/1.4818285}{\textcolor{blue}{http://dx.doi.org/10.1063/1.4818285}}

\bibitem {}
Hazar A. \.Ilhan, Mert Do\v{g}ar, Meri\c{c} \"Ozcan (2013),
\newblock {\em Fast autofocusing in digital holography using scaled holograms},
\newblock Optics Communications, Volume: 287, 81--84
\newblock \newline{} \href{http://dx.doi.org/10.1016/j.optcom.2012.09.036}{\textcolor{blue}{http://dx.doi.org/10.1016/j.optcom.2012.09.036}}

\bibitem {}
Hazar A. \.Ilhan, Mert Do\v{g}ar, Meri\c{c} \"Ozcan (2013),
\newblock {\em Autofocusing in digital holography},
\newblock In Proceedings of SPIE 8644, Practical Holography XXVII: Materials and Applications, 86440C, San Francisco
\newblock \newline{} \href{http://dx.doi.org/10.1117/12.2002038}{\textcolor{blue}{http://dx.doi.org/10.1117/12.2002038}}

\bibitem {}
Mert Do\v{g}ar, Hazar A. \.Ilhan, Meri\c{c} \"Ozcan \"Ozcan (2013),
\newblock {\em Real-time reconstruction of digital holograms with GPU},
\newblock In Proceedings of SPIE 8644, Practical Holography XXVII: Materials and Applications, 86440B, San Francisco
\newblock \newline{} \href{http://dx.doi.org/10.1117/12.2002036}{\textcolor{blue}{http://dx.doi.org/10.1117/12.2002036}}

\end{thebibliography}


\end{document}

